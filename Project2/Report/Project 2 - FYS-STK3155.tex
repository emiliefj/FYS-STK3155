\documentclass[11pt]{article}

    \usepackage[breakable]{tcolorbox}
    \usepackage{parskip} % Stop auto-indenting (to mimic markdown behaviour)
    
    \usepackage{iftex}
    \ifPDFTeX
    	\usepackage[T1]{fontenc}
    	\usepackage{mathpazo}
    \else
    	\usepackage{fontspec}
    \fi

    % Basic figure setup, for now with no caption control since it's done
    % automatically by Pandoc (which extracts ![](path) syntax from Markdown).
    \usepackage{graphicx}
    % Maintain compatibility with old templates. Remove in nbconvert 6.0
    \let\Oldincludegraphics\includegraphics
    % Ensure that by default, figures have no caption (until we provide a
    % proper Figure object with a Caption API and a way to capture that
    % in the conversion process - todo).
    \usepackage{caption}
    \DeclareCaptionFormat{nocaption}{}
    \captionsetup{format=nocaption,aboveskip=0pt,belowskip=0pt}

    \usepackage{float}
    \floatplacement{figure}{H} % forces figures to be placed at the correct location
    \usepackage{xcolor} % Allow colors to be defined
    \usepackage{enumerate} % Needed for markdown enumerations to work
    \usepackage{geometry} % Used to adjust the document margins
    \usepackage{amsmath} % Equations
    \usepackage{amssymb} % Equations
    \usepackage{textcomp} % defines textquotesingle
    % Hack from http://tex.stackexchange.com/a/47451/13684:
    \AtBeginDocument{%
        \def\PYZsq{\textquotesingle}% Upright quotes in Pygmentized code
    }
    \usepackage{upquote} % Upright quotes for verbatim code
    \usepackage{eurosym} % defines \euro
    \usepackage[mathletters]{ucs} % Extended unicode (utf-8) support
    \usepackage{fancyvrb} % verbatim replacement that allows latex
    \usepackage{grffile} % extends the file name processing of package graphics 
                         % to support a larger range
    \makeatletter % fix for old versions of grffile with XeLaTeX
    \@ifpackagelater{grffile}{2019/11/01}
    {
      % Do nothing on new versions
    }
    {
      \def\Gread@@xetex#1{%
        \IfFileExists{"\Gin@base".bb}%
        {\Gread@eps{\Gin@base.bb}}%
        {\Gread@@xetex@aux#1}%
      }
    }
    \makeatother
    \usepackage[Export]{adjustbox} % Used to constrain images to a maximum size
    \adjustboxset{max size={0.9\linewidth}{0.9\paperheight}}

    % The hyperref package gives us a pdf with properly built
    % internal navigation ('pdf bookmarks' for the table of contents,
    % internal cross-reference links, web links for URLs, etc.)
    \usepackage{hyperref}
    % The default LaTeX title has an obnoxious amount of whitespace. By default,
    % titling removes some of it. It also provides customization options.
    \usepackage{titling}
    \usepackage{longtable} % longtable support required by pandoc >1.10
    \usepackage{booktabs}  % table support for pandoc > 1.12.2
    \usepackage[inline]{enumitem} % IRkernel/repr support (it uses the enumerate* environment)
    \usepackage[normalem]{ulem} % ulem is needed to support strikethroughs (\sout)
                                % normalem makes italics be italics, not underlines
    \usepackage{mathrsfs}
    

    
    % Colors for the hyperref package
    \definecolor{urlcolor}{rgb}{0,.145,.698}
    \definecolor{linkcolor}{rgb}{.71,0.21,0.01}
    \definecolor{citecolor}{rgb}{.12,.54,.11}

    % ANSI colors
    \definecolor{ansi-black}{HTML}{3E424D}
    \definecolor{ansi-black-intense}{HTML}{282C36}
    \definecolor{ansi-red}{HTML}{E75C58}
    \definecolor{ansi-red-intense}{HTML}{B22B31}
    \definecolor{ansi-green}{HTML}{00A250}
    \definecolor{ansi-green-intense}{HTML}{007427}
    \definecolor{ansi-yellow}{HTML}{DDB62B}
    \definecolor{ansi-yellow-intense}{HTML}{B27D12}
    \definecolor{ansi-blue}{HTML}{208FFB}
    \definecolor{ansi-blue-intense}{HTML}{0065CA}
    \definecolor{ansi-magenta}{HTML}{D160C4}
    \definecolor{ansi-magenta-intense}{HTML}{A03196}
    \definecolor{ansi-cyan}{HTML}{60C6C8}
    \definecolor{ansi-cyan-intense}{HTML}{258F8F}
    \definecolor{ansi-white}{HTML}{C5C1B4}
    \definecolor{ansi-white-intense}{HTML}{A1A6B2}
    \definecolor{ansi-default-inverse-fg}{HTML}{FFFFFF}
    \definecolor{ansi-default-inverse-bg}{HTML}{000000}

    % common color for the border for error outputs.
    \definecolor{outerrorbackground}{HTML}{FFDFDF}

    % commands and environments needed by pandoc snippets
    % extracted from the output of `pandoc -s`
    \providecommand{\tightlist}{%
      \setlength{\itemsep}{0pt}\setlength{\parskip}{0pt}}
    \DefineVerbatimEnvironment{Highlighting}{Verbatim}{commandchars=\\\{\}}
    % Add ',fontsize=\small' for more characters per line
    \newenvironment{Shaded}{}{}
    \newcommand{\KeywordTok}[1]{\textcolor[rgb]{0.00,0.44,0.13}{\textbf{{#1}}}}
    \newcommand{\DataTypeTok}[1]{\textcolor[rgb]{0.56,0.13,0.00}{{#1}}}
    \newcommand{\DecValTok}[1]{\textcolor[rgb]{0.25,0.63,0.44}{{#1}}}
    \newcommand{\BaseNTok}[1]{\textcolor[rgb]{0.25,0.63,0.44}{{#1}}}
    \newcommand{\FloatTok}[1]{\textcolor[rgb]{0.25,0.63,0.44}{{#1}}}
    \newcommand{\CharTok}[1]{\textcolor[rgb]{0.25,0.44,0.63}{{#1}}}
    \newcommand{\StringTok}[1]{\textcolor[rgb]{0.25,0.44,0.63}{{#1}}}
    \newcommand{\CommentTok}[1]{\textcolor[rgb]{0.38,0.63,0.69}{\textit{{#1}}}}
    \newcommand{\OtherTok}[1]{\textcolor[rgb]{0.00,0.44,0.13}{{#1}}}
    \newcommand{\AlertTok}[1]{\textcolor[rgb]{1.00,0.00,0.00}{\textbf{{#1}}}}
    \newcommand{\FunctionTok}[1]{\textcolor[rgb]{0.02,0.16,0.49}{{#1}}}
    \newcommand{\RegionMarkerTok}[1]{{#1}}
    \newcommand{\ErrorTok}[1]{\textcolor[rgb]{1.00,0.00,0.00}{\textbf{{#1}}}}
    \newcommand{\NormalTok}[1]{{#1}}
    
    % Additional commands for more recent versions of Pandoc
    \newcommand{\ConstantTok}[1]{\textcolor[rgb]{0.53,0.00,0.00}{{#1}}}
    \newcommand{\SpecialCharTok}[1]{\textcolor[rgb]{0.25,0.44,0.63}{{#1}}}
    \newcommand{\VerbatimStringTok}[1]{\textcolor[rgb]{0.25,0.44,0.63}{{#1}}}
    \newcommand{\SpecialStringTok}[1]{\textcolor[rgb]{0.73,0.40,0.53}{{#1}}}
    \newcommand{\ImportTok}[1]{{#1}}
    \newcommand{\DocumentationTok}[1]{\textcolor[rgb]{0.73,0.13,0.13}{\textit{{#1}}}}
    \newcommand{\AnnotationTok}[1]{\textcolor[rgb]{0.38,0.63,0.69}{\textbf{\textit{{#1}}}}}
    \newcommand{\CommentVarTok}[1]{\textcolor[rgb]{0.38,0.63,0.69}{\textbf{\textit{{#1}}}}}
    \newcommand{\VariableTok}[1]{\textcolor[rgb]{0.10,0.09,0.49}{{#1}}}
    \newcommand{\ControlFlowTok}[1]{\textcolor[rgb]{0.00,0.44,0.13}{\textbf{{#1}}}}
    \newcommand{\OperatorTok}[1]{\textcolor[rgb]{0.40,0.40,0.40}{{#1}}}
    \newcommand{\BuiltInTok}[1]{{#1}}
    \newcommand{\ExtensionTok}[1]{{#1}}
    \newcommand{\PreprocessorTok}[1]{\textcolor[rgb]{0.74,0.48,0.00}{{#1}}}
    \newcommand{\AttributeTok}[1]{\textcolor[rgb]{0.49,0.56,0.16}{{#1}}}
    \newcommand{\InformationTok}[1]{\textcolor[rgb]{0.38,0.63,0.69}{\textbf{\textit{{#1}}}}}
    \newcommand{\WarningTok}[1]{\textcolor[rgb]{0.38,0.63,0.69}{\textbf{\textit{{#1}}}}}
    
    
    % Define a nice break command that doesn't care if a line doesn't already
    % exist.
    \def\br{\hspace*{\fill} \\* }
    % Math Jax compatibility definitions
    \def\gt{>}
    \def\lt{<}
    \let\Oldtex\TeX
    \let\Oldlatex\LaTeX
    \renewcommand{\TeX}{\textrm{\Oldtex}}
    \renewcommand{\LaTeX}{\textrm{\Oldlatex}}
    % Document parameters
    % Document title
    \title{Project 2 - FYS-STK3155}
    
    
    
    
    
% Pygments definitions
\makeatletter
\def\PY@reset{\let\PY@it=\relax \let\PY@bf=\relax%
    \let\PY@ul=\relax \let\PY@tc=\relax%
    \let\PY@bc=\relax \let\PY@ff=\relax}
\def\PY@tok#1{\csname PY@tok@#1\endcsname}
\def\PY@toks#1+{\ifx\relax#1\empty\else%
    \PY@tok{#1}\expandafter\PY@toks\fi}
\def\PY@do#1{\PY@bc{\PY@tc{\PY@ul{%
    \PY@it{\PY@bf{\PY@ff{#1}}}}}}}
\def\PY#1#2{\PY@reset\PY@toks#1+\relax+\PY@do{#2}}

\expandafter\def\csname PY@tok@w\endcsname{\def\PY@tc##1{\textcolor[rgb]{0.73,0.73,0.73}{##1}}}
\expandafter\def\csname PY@tok@c\endcsname{\let\PY@it=\textit\def\PY@tc##1{\textcolor[rgb]{0.25,0.50,0.50}{##1}}}
\expandafter\def\csname PY@tok@cp\endcsname{\def\PY@tc##1{\textcolor[rgb]{0.74,0.48,0.00}{##1}}}
\expandafter\def\csname PY@tok@k\endcsname{\let\PY@bf=\textbf\def\PY@tc##1{\textcolor[rgb]{0.00,0.50,0.00}{##1}}}
\expandafter\def\csname PY@tok@kp\endcsname{\def\PY@tc##1{\textcolor[rgb]{0.00,0.50,0.00}{##1}}}
\expandafter\def\csname PY@tok@kt\endcsname{\def\PY@tc##1{\textcolor[rgb]{0.69,0.00,0.25}{##1}}}
\expandafter\def\csname PY@tok@o\endcsname{\def\PY@tc##1{\textcolor[rgb]{0.40,0.40,0.40}{##1}}}
\expandafter\def\csname PY@tok@ow\endcsname{\let\PY@bf=\textbf\def\PY@tc##1{\textcolor[rgb]{0.67,0.13,1.00}{##1}}}
\expandafter\def\csname PY@tok@nb\endcsname{\def\PY@tc##1{\textcolor[rgb]{0.00,0.50,0.00}{##1}}}
\expandafter\def\csname PY@tok@nf\endcsname{\def\PY@tc##1{\textcolor[rgb]{0.00,0.00,1.00}{##1}}}
\expandafter\def\csname PY@tok@nc\endcsname{\let\PY@bf=\textbf\def\PY@tc##1{\textcolor[rgb]{0.00,0.00,1.00}{##1}}}
\expandafter\def\csname PY@tok@nn\endcsname{\let\PY@bf=\textbf\def\PY@tc##1{\textcolor[rgb]{0.00,0.00,1.00}{##1}}}
\expandafter\def\csname PY@tok@ne\endcsname{\let\PY@bf=\textbf\def\PY@tc##1{\textcolor[rgb]{0.82,0.25,0.23}{##1}}}
\expandafter\def\csname PY@tok@nv\endcsname{\def\PY@tc##1{\textcolor[rgb]{0.10,0.09,0.49}{##1}}}
\expandafter\def\csname PY@tok@no\endcsname{\def\PY@tc##1{\textcolor[rgb]{0.53,0.00,0.00}{##1}}}
\expandafter\def\csname PY@tok@nl\endcsname{\def\PY@tc##1{\textcolor[rgb]{0.63,0.63,0.00}{##1}}}
\expandafter\def\csname PY@tok@ni\endcsname{\let\PY@bf=\textbf\def\PY@tc##1{\textcolor[rgb]{0.60,0.60,0.60}{##1}}}
\expandafter\def\csname PY@tok@na\endcsname{\def\PY@tc##1{\textcolor[rgb]{0.49,0.56,0.16}{##1}}}
\expandafter\def\csname PY@tok@nt\endcsname{\let\PY@bf=\textbf\def\PY@tc##1{\textcolor[rgb]{0.00,0.50,0.00}{##1}}}
\expandafter\def\csname PY@tok@nd\endcsname{\def\PY@tc##1{\textcolor[rgb]{0.67,0.13,1.00}{##1}}}
\expandafter\def\csname PY@tok@s\endcsname{\def\PY@tc##1{\textcolor[rgb]{0.73,0.13,0.13}{##1}}}
\expandafter\def\csname PY@tok@sd\endcsname{\let\PY@it=\textit\def\PY@tc##1{\textcolor[rgb]{0.73,0.13,0.13}{##1}}}
\expandafter\def\csname PY@tok@si\endcsname{\let\PY@bf=\textbf\def\PY@tc##1{\textcolor[rgb]{0.73,0.40,0.53}{##1}}}
\expandafter\def\csname PY@tok@se\endcsname{\let\PY@bf=\textbf\def\PY@tc##1{\textcolor[rgb]{0.73,0.40,0.13}{##1}}}
\expandafter\def\csname PY@tok@sr\endcsname{\def\PY@tc##1{\textcolor[rgb]{0.73,0.40,0.53}{##1}}}
\expandafter\def\csname PY@tok@ss\endcsname{\def\PY@tc##1{\textcolor[rgb]{0.10,0.09,0.49}{##1}}}
\expandafter\def\csname PY@tok@sx\endcsname{\def\PY@tc##1{\textcolor[rgb]{0.00,0.50,0.00}{##1}}}
\expandafter\def\csname PY@tok@m\endcsname{\def\PY@tc##1{\textcolor[rgb]{0.40,0.40,0.40}{##1}}}
\expandafter\def\csname PY@tok@gh\endcsname{\let\PY@bf=\textbf\def\PY@tc##1{\textcolor[rgb]{0.00,0.00,0.50}{##1}}}
\expandafter\def\csname PY@tok@gu\endcsname{\let\PY@bf=\textbf\def\PY@tc##1{\textcolor[rgb]{0.50,0.00,0.50}{##1}}}
\expandafter\def\csname PY@tok@gd\endcsname{\def\PY@tc##1{\textcolor[rgb]{0.63,0.00,0.00}{##1}}}
\expandafter\def\csname PY@tok@gi\endcsname{\def\PY@tc##1{\textcolor[rgb]{0.00,0.63,0.00}{##1}}}
\expandafter\def\csname PY@tok@gr\endcsname{\def\PY@tc##1{\textcolor[rgb]{1.00,0.00,0.00}{##1}}}
\expandafter\def\csname PY@tok@ge\endcsname{\let\PY@it=\textit}
\expandafter\def\csname PY@tok@gs\endcsname{\let\PY@bf=\textbf}
\expandafter\def\csname PY@tok@gp\endcsname{\let\PY@bf=\textbf\def\PY@tc##1{\textcolor[rgb]{0.00,0.00,0.50}{##1}}}
\expandafter\def\csname PY@tok@go\endcsname{\def\PY@tc##1{\textcolor[rgb]{0.53,0.53,0.53}{##1}}}
\expandafter\def\csname PY@tok@gt\endcsname{\def\PY@tc##1{\textcolor[rgb]{0.00,0.27,0.87}{##1}}}
\expandafter\def\csname PY@tok@err\endcsname{\def\PY@bc##1{\setlength{\fboxsep}{0pt}\fcolorbox[rgb]{1.00,0.00,0.00}{1,1,1}{\strut ##1}}}
\expandafter\def\csname PY@tok@kc\endcsname{\let\PY@bf=\textbf\def\PY@tc##1{\textcolor[rgb]{0.00,0.50,0.00}{##1}}}
\expandafter\def\csname PY@tok@kd\endcsname{\let\PY@bf=\textbf\def\PY@tc##1{\textcolor[rgb]{0.00,0.50,0.00}{##1}}}
\expandafter\def\csname PY@tok@kn\endcsname{\let\PY@bf=\textbf\def\PY@tc##1{\textcolor[rgb]{0.00,0.50,0.00}{##1}}}
\expandafter\def\csname PY@tok@kr\endcsname{\let\PY@bf=\textbf\def\PY@tc##1{\textcolor[rgb]{0.00,0.50,0.00}{##1}}}
\expandafter\def\csname PY@tok@bp\endcsname{\def\PY@tc##1{\textcolor[rgb]{0.00,0.50,0.00}{##1}}}
\expandafter\def\csname PY@tok@fm\endcsname{\def\PY@tc##1{\textcolor[rgb]{0.00,0.00,1.00}{##1}}}
\expandafter\def\csname PY@tok@vc\endcsname{\def\PY@tc##1{\textcolor[rgb]{0.10,0.09,0.49}{##1}}}
\expandafter\def\csname PY@tok@vg\endcsname{\def\PY@tc##1{\textcolor[rgb]{0.10,0.09,0.49}{##1}}}
\expandafter\def\csname PY@tok@vi\endcsname{\def\PY@tc##1{\textcolor[rgb]{0.10,0.09,0.49}{##1}}}
\expandafter\def\csname PY@tok@vm\endcsname{\def\PY@tc##1{\textcolor[rgb]{0.10,0.09,0.49}{##1}}}
\expandafter\def\csname PY@tok@sa\endcsname{\def\PY@tc##1{\textcolor[rgb]{0.73,0.13,0.13}{##1}}}
\expandafter\def\csname PY@tok@sb\endcsname{\def\PY@tc##1{\textcolor[rgb]{0.73,0.13,0.13}{##1}}}
\expandafter\def\csname PY@tok@sc\endcsname{\def\PY@tc##1{\textcolor[rgb]{0.73,0.13,0.13}{##1}}}
\expandafter\def\csname PY@tok@dl\endcsname{\def\PY@tc##1{\textcolor[rgb]{0.73,0.13,0.13}{##1}}}
\expandafter\def\csname PY@tok@s2\endcsname{\def\PY@tc##1{\textcolor[rgb]{0.73,0.13,0.13}{##1}}}
\expandafter\def\csname PY@tok@sh\endcsname{\def\PY@tc##1{\textcolor[rgb]{0.73,0.13,0.13}{##1}}}
\expandafter\def\csname PY@tok@s1\endcsname{\def\PY@tc##1{\textcolor[rgb]{0.73,0.13,0.13}{##1}}}
\expandafter\def\csname PY@tok@mb\endcsname{\def\PY@tc##1{\textcolor[rgb]{0.40,0.40,0.40}{##1}}}
\expandafter\def\csname PY@tok@mf\endcsname{\def\PY@tc##1{\textcolor[rgb]{0.40,0.40,0.40}{##1}}}
\expandafter\def\csname PY@tok@mh\endcsname{\def\PY@tc##1{\textcolor[rgb]{0.40,0.40,0.40}{##1}}}
\expandafter\def\csname PY@tok@mi\endcsname{\def\PY@tc##1{\textcolor[rgb]{0.40,0.40,0.40}{##1}}}
\expandafter\def\csname PY@tok@il\endcsname{\def\PY@tc##1{\textcolor[rgb]{0.40,0.40,0.40}{##1}}}
\expandafter\def\csname PY@tok@mo\endcsname{\def\PY@tc##1{\textcolor[rgb]{0.40,0.40,0.40}{##1}}}
\expandafter\def\csname PY@tok@ch\endcsname{\let\PY@it=\textit\def\PY@tc##1{\textcolor[rgb]{0.25,0.50,0.50}{##1}}}
\expandafter\def\csname PY@tok@cm\endcsname{\let\PY@it=\textit\def\PY@tc##1{\textcolor[rgb]{0.25,0.50,0.50}{##1}}}
\expandafter\def\csname PY@tok@cpf\endcsname{\let\PY@it=\textit\def\PY@tc##1{\textcolor[rgb]{0.25,0.50,0.50}{##1}}}
\expandafter\def\csname PY@tok@c1\endcsname{\let\PY@it=\textit\def\PY@tc##1{\textcolor[rgb]{0.25,0.50,0.50}{##1}}}
\expandafter\def\csname PY@tok@cs\endcsname{\let\PY@it=\textit\def\PY@tc##1{\textcolor[rgb]{0.25,0.50,0.50}{##1}}}

\def\PYZbs{\char`\\}
\def\PYZus{\char`\_}
\def\PYZob{\char`\{}
\def\PYZcb{\char`\}}
\def\PYZca{\char`\^}
\def\PYZam{\char`\&}
\def\PYZlt{\char`\<}
\def\PYZgt{\char`\>}
\def\PYZsh{\char`\#}
\def\PYZpc{\char`\%}
\def\PYZdl{\char`\$}
\def\PYZhy{\char`\-}
\def\PYZsq{\char`\'}
\def\PYZdq{\char`\"}
\def\PYZti{\char`\~}
% for compatibility with earlier versions
\def\PYZat{@}
\def\PYZlb{[}
\def\PYZrb{]}
\makeatother


    % For linebreaks inside Verbatim environment from package fancyvrb. 
    \makeatletter
        \newbox\Wrappedcontinuationbox 
        \newbox\Wrappedvisiblespacebox 
        \newcommand*\Wrappedvisiblespace {\textcolor{red}{\textvisiblespace}} 
        \newcommand*\Wrappedcontinuationsymbol {\textcolor{red}{\llap{\tiny$\m@th\hookrightarrow$}}} 
        \newcommand*\Wrappedcontinuationindent {3ex } 
        \newcommand*\Wrappedafterbreak {\kern\Wrappedcontinuationindent\copy\Wrappedcontinuationbox} 
        % Take advantage of the already applied Pygments mark-up to insert 
        % potential linebreaks for TeX processing. 
        %        {, <, #, %, $, ' and ": go to next line. 
        %        _, }, ^, &, >, - and ~: stay at end of broken line. 
        % Use of \textquotesingle for straight quote. 
        \newcommand*\Wrappedbreaksatspecials {% 
            \def\PYGZus{\discretionary{\char`\_}{\Wrappedafterbreak}{\char`\_}}% 
            \def\PYGZob{\discretionary{}{\Wrappedafterbreak\char`\{}{\char`\{}}% 
            \def\PYGZcb{\discretionary{\char`\}}{\Wrappedafterbreak}{\char`\}}}% 
            \def\PYGZca{\discretionary{\char`\^}{\Wrappedafterbreak}{\char`\^}}% 
            \def\PYGZam{\discretionary{\char`\&}{\Wrappedafterbreak}{\char`\&}}% 
            \def\PYGZlt{\discretionary{}{\Wrappedafterbreak\char`\<}{\char`\<}}% 
            \def\PYGZgt{\discretionary{\char`\>}{\Wrappedafterbreak}{\char`\>}}% 
            \def\PYGZsh{\discretionary{}{\Wrappedafterbreak\char`\#}{\char`\#}}% 
            \def\PYGZpc{\discretionary{}{\Wrappedafterbreak\char`\%}{\char`\%}}% 
            \def\PYGZdl{\discretionary{}{\Wrappedafterbreak\char`\$}{\char`\$}}% 
            \def\PYGZhy{\discretionary{\char`\-}{\Wrappedafterbreak}{\char`\-}}% 
            \def\PYGZsq{\discretionary{}{\Wrappedafterbreak\textquotesingle}{\textquotesingle}}% 
            \def\PYGZdq{\discretionary{}{\Wrappedafterbreak\char`\"}{\char`\"}}% 
            \def\PYGZti{\discretionary{\char`\~}{\Wrappedafterbreak}{\char`\~}}% 
        } 
        % Some characters . , ; ? ! / are not pygmentized. 
        % This macro makes them "active" and they will insert potential linebreaks 
        \newcommand*\Wrappedbreaksatpunct {% 
            \lccode`\~`\.\lowercase{\def~}{\discretionary{\hbox{\char`\.}}{\Wrappedafterbreak}{\hbox{\char`\.}}}% 
            \lccode`\~`\,\lowercase{\def~}{\discretionary{\hbox{\char`\,}}{\Wrappedafterbreak}{\hbox{\char`\,}}}% 
            \lccode`\~`\;\lowercase{\def~}{\discretionary{\hbox{\char`\;}}{\Wrappedafterbreak}{\hbox{\char`\;}}}% 
            \lccode`\~`\:\lowercase{\def~}{\discretionary{\hbox{\char`\:}}{\Wrappedafterbreak}{\hbox{\char`\:}}}% 
            \lccode`\~`\?\lowercase{\def~}{\discretionary{\hbox{\char`\?}}{\Wrappedafterbreak}{\hbox{\char`\?}}}% 
            \lccode`\~`\!\lowercase{\def~}{\discretionary{\hbox{\char`\!}}{\Wrappedafterbreak}{\hbox{\char`\!}}}% 
            \lccode`\~`\/\lowercase{\def~}{\discretionary{\hbox{\char`\/}}{\Wrappedafterbreak}{\hbox{\char`\/}}}% 
            \catcode`\.\active
            \catcode`\,\active 
            \catcode`\;\active
            \catcode`\:\active
            \catcode`\?\active
            \catcode`\!\active
            \catcode`\/\active 
            \lccode`\~`\~ 	
        }
    \makeatother

    \let\OriginalVerbatim=\Verbatim
    \makeatletter
    \renewcommand{\Verbatim}[1][1]{%
        %\parskip\z@skip
        \sbox\Wrappedcontinuationbox {\Wrappedcontinuationsymbol}%
        \sbox\Wrappedvisiblespacebox {\FV@SetupFont\Wrappedvisiblespace}%
        \def\FancyVerbFormatLine ##1{\hsize\linewidth
            \vtop{\raggedright\hyphenpenalty\z@\exhyphenpenalty\z@
                \doublehyphendemerits\z@\finalhyphendemerits\z@
                \strut ##1\strut}%
        }%
        % If the linebreak is at a space, the latter will be displayed as visible
        % space at end of first line, and a continuation symbol starts next line.
        % Stretch/shrink are however usually zero for typewriter font.
        \def\FV@Space {%
            \nobreak\hskip\z@ plus\fontdimen3\font minus\fontdimen4\font
            \discretionary{\copy\Wrappedvisiblespacebox}{\Wrappedafterbreak}
            {\kern\fontdimen2\font}%
        }%
        
        % Allow breaks at special characters using \PYG... macros.
        \Wrappedbreaksatspecials
        % Breaks at punctuation characters . , ; ? ! and / need catcode=\active 	
        \OriginalVerbatim[#1,codes*=\Wrappedbreaksatpunct]%
    }
    \makeatother

    % Exact colors from NB
    \definecolor{incolor}{HTML}{303F9F}
    \definecolor{outcolor}{HTML}{D84315}
    \definecolor{cellborder}{HTML}{CFCFCF}
    \definecolor{cellbackground}{HTML}{F7F7F7}
    
    % prompt
    \makeatletter
    \newcommand{\boxspacing}{\kern\kvtcb@left@rule\kern\kvtcb@boxsep}
    \makeatother
    \newcommand{\prompt}[4]{
        {\ttfamily\llap{{\color{#2}[#3]:\hspace{3pt}#4}}\vspace{-\baselineskip}}
    }
    

    
    % Prevent overflowing lines due to hard-to-break entities
    \sloppy 
    % Setup hyperref package
    \hypersetup{
      breaklinks=true,  % so long urls are correctly broken across lines
      colorlinks=true,
      urlcolor=urlcolor,
      linkcolor=linkcolor,
      citecolor=citecolor,
      }
    % Slightly bigger margins than the latex defaults
    
    \geometry{verbose,tmargin=1in,bmargin=1in,lmargin=1in,rmargin=1in}
    
    

\begin{document}
    
    \maketitle
    
    

    
    \hypertarget{abstract}{%
\section{Abstract}\label{abstract}}

    We have explored

    \hypertarget{introduction}{%
\section{Introduction}\label{introduction}}

    

    \hypertarget{data}{%
\section{Data}\label{data}}

\hypertarget{franke-function-and-terrain-data}{%
\subsection{Franke Function and Terrain
Data}\label{franke-function-and-terrain-data}}

We will revisit these datasets from project one. For further description
refer to the report for project 1 {[}1{]}

    \hypertarget{mnist-handwritten-numbers}{%
\subsection{MNIST Handwritten Numbers}\label{mnist-handwritten-numbers}}

For our classification problem we will be using the MNIST database of
handwritten numbers, or more specifically a subset of this available as
the digits dataset in scikit-learn{[}2{]}.

This dataset contains pixel images of handwritten numbers. In total the
digits dataset contains 1797 8x8 images. Meaning we have a dataset
consisting of 1797 data points, each with 64 descriptors or inputs, and
one output, the label.

For a plot of one such data point, see figure 1.

    \begin{center}
    \adjustimage{max size={0.9\linewidth}{0.9\paperheight}}{output_6_0.png}
    \end{center}
    { \hspace*{\fill} \\}
    
    \begin{Verbatim}[commandchars=\\\{\}]
Figure 1: Plot of one of the entries in the digits dataset showing a number 0.
    \end{Verbatim}

    We will use this dataset to fit and test our classification models. To
prepare the data we split it into a training set and a test set. I use
\(\texttt{scikit\-learn}\)'s \(\texttt{train\_test\_split}()\) for this and
an 80-20 split.


    We want to make sure that the different classes, i.e.~the labels are
similarly distributed among the test set and the training test. If for
instance no occurrences of the label `3' is found in the training set,
models trained on this set are expected to perform poorly on identifying
such digits in the test set. To avoid such issues we set the
\(\texttt{stratify}\) parameter in \(\texttt{train\_test\_split()}\) to
the target variable. The result is seen in figure 5, and we can see that
the results are satisfactory; the labels appear to be very evenly
distributed.


    \begin{center}
    \adjustimage{max size={0.9\linewidth}{0.9\paperheight}}{output_10_0.png}
    \end{center}
    { \hspace*{\fill} \\}
    
    \begin{Verbatim}[commandchars=\\\{\}]
Figure 2: Histogram of label distribution in training and test set.
    \end{Verbatim}

    Figure 2 shows the distribution of the different labels int the test and
training sets. We see that the labels seem to be very evenly
distributed.

    \hypertarget{methods}{%
\section{Methods}\label{methods}}

For descriptions of ordinary least squares and ridge regression, please
refer to my report for project 1{[}1{]}. \#\# Performance Measures For a
description of the mean square error and \(r^2\) please again consult
the report of project 1{[}1{]}.

\hypertarget{accuracy}{%
\subsubsection{Accuracy}\label{accuracy}}

We use the accuracy score to measure the performance of our
classification methods. This measure is given by the number of correctly
guessed targets \(t_i\) divided by the total number of targets, that is

\[
\text{Accuracy} = \frac{\sum_{i=1}^n I(t_i = y_i)}{n} ,
\]

where \(I\) is the indicator function, which takes the value \(1\) if
\(t_i = y_i\) and \(0\) otherwise (for a binary classification problem).
Here \(t_i\) represents the target and \(y_i\) is the prediction. The
number of targets is given by \(n\).

    \hypertarget{regression-methods}{%
\subsection{Regression Methods}\label{regression-methods}}

For description of regular linear regression see Methods section in
project 1 {[}1{]}.

\hypertarget{logistic-regression}{%
\subsubsection{Logistic Regression}\label{logistic-regression}}

Logistic regression is a machine learning algorithm typically used for
classification problems. It is most typically applied in binary
classification problems, e.g.~true/false, yes/no, positive/negative, but
it can be extended to problems with multiple classes, called multinomial
logistic regression.

I logistic regression a logistic function is applied to model the
dependent variable. Such a logistic function takes in any input value,
and will always return a value between zero and one. This is often
viewed as transforming the input and outputting a probability value. In
classification we are interested in discrete output. Some functions,
like the step function (see figure number) produces a binary zero or
one, while we for other functions, like the sigmoid translate the output
values according to a cutoff, e.g.~output above or equal 0.5 gets coded
as a one.

The classic function used to transform the input in logistic regression
is the sigmoid, see the section on activation functions and figure 4.

The general form of logistic regression is

\[
\hat{y_k} = g(\beta X_k),
\]

where \(g(z)\) is the sigmoid and \(\beta\) are weights fit to the
inputs in training the model.

While we in linear regression used the mean square error as the cost
function, this cost function will we non-convex for the logistic case.
This makes finding the minimum difficult, and for this reason we cannot
use it as our cost function for logistic regression. To find a cost
function we can use the maximum likelihood estimator. In the binary case
this has the form,


\begin{align*}
P((x_i,y_i)|\hat{\beta})& = \prod_{i=1}^n \left[p(y_i=1|x_i,\hat{\beta})\right]^{y_i}\left[1-p(y_i=1|x_i,\hat{\beta}))\right]^{1-y_i}\nonumber \\
\end{align*}


which by taking the logarithm and reordering leads to the cross-entropy

\[
\mathcal{C}(\hat{\beta})=-\sum_{i=1}^n  \left(y_i(\boldsymbol{\beta}x_i) -\log{(1+\exp{(\boldsymbol{\beta}x_i)})}\right).
\] The cross-entropy will be a convex function of the weights
\(\hat{\beta}\). To train our model we want to minimixe the derivative
of the cost function with respect to the weights \(\beta\).

    \hypertarget{multinomial-logistic-regression9}{%
\paragraph{Multinomial Logistic
Regression{[}9{]}}\label{multinomial-logistic-regression9}}

In multinomial logistic regression, also known as softmax regression we
use the same general procedure as in binomial logistic regression, but
now the output of the model must be coded into discrete values using
several ranges. If for example we are interested in classifying our
input to one of four classes A, B, C and D we can code an output between
0 and 0.25 to A, an output between 0.25 and 0.5 to B and so forth.

The \emph{Softmax} function is the preferred output or activation
function for multinomial logistic regression. For a classification
problem with \(K\) classes, the softmax gives the probability for each
class.

Using maximum likelihood and a set of \(K\) classes \(k_i\) we get

\[
\prod_{i=1}^N\prod_{k=1}^K P(y_i=k|x_i,\beta)
\]

with the probabilities given by the softmax function

\[
P(y_i=k|x_i,\beta) = \frac{\exp(\beta^{(k)}x_i)}{\sum_{j=1}^K \exp(\beta^{(j)}x_i)}
\]

for each training data point \(i\). Here \(\beta^{(k)}\) are the weights
of the model. We see from the form of the softmax that it is the
probability of class \(k_i\) divided by the probability of all the other
classes.

We want to maximize this, or equivalently, minimize its logarithm

\[
C(\beta) = -\sum_{i=1}^N\sum_{k=1}^K 1\{y_i=k\}\log\frac{\exp(\beta^{(k)T}x_i)}{\sum_{j=1}^K \exp(\beta^{(j)T}x_i)}
\]

which is our cost function. Here \(\{\texttt{statement}\}\) is \(1\) if
\(\texttt{statement}\) is \(\texttt{True}\) and \(0\) otherwise.

To train our model we want to minimize the cost function. To do this we
need the gradient, which is given by

\[
\nabla_{\beta^k} C(\beta) = -\sum_{i=1}^Nx_i\bigg(1\{y_i=k\}-P(y_i=k|x_i,\beta)\bigg).
\]

A method like stochastic gradient descent or newton's method can be used
together with this to train the model and find optimal weights
\(\beta\).

    \hypertarget{stochastic-gradient-descent}{%
\subsection{Stochastic Gradient
Descent}\label{stochastic-gradient-descent}}

Stochastic Gradient Descent (SGD) is a variation of the gradient descent
method. There are many different variations of gradient descent but the
common denominator is that they work to find the minima of a function
(typically the cost function) by iteratively moving in the direction of
steepest descent.

In other words, if we want to find the minimum of a function
\(F(\mathbf{x})\), we should move in the direction of the negative
gradient \(-\nabla F(\mathbf{x})\).

For a (typically small) stepsize \(\gamma_k > 0\) known as the step
size, or learning rate we then have

\[
\mathbf{x}_{k+1} = \mathbf{x}_k - \gamma_k \nabla F(\mathbf{x}_k),
\]

An initial guess is made at the first step.

For a convex cost function and a sufficiently small stepsize
\(\gamma_k\) this method can be used to find the global minimum of said
cost function.

One drawback of this method is that computing the gradient for large
datasets can be very computationally expensive. This is where SGD and
its variations comes in. Instead of calculating the gradiant for all
datapoints, SGD chooses a subset of the data to calculate the gradient
on. The method divides the dataset into a set of \(N/M\) so-called
mini-batches, and for each step the gradient is calculated on one of
these mini-batches. These mini-batches are denoted by \(B_k\) where
\(k=1,\cdots,N/M\). \(N\) is here the size of the training data, and
\(M\) is the size of each mini-batch.

Rewriting the gradient descent method in terms of a cost function
\(C(\mathbf{\beta})\), we get

\[
\mathbf{\beta}_{k+1} = \mathbf{\beta}_k - \gamma_k \nabla_{\beta} C(\mathbf{x}_k,\mathbf{\beta}_k),
\]

Taking the gradient with respect to one mini-batch per step we get

\[
\beta_{k+1} = \beta_k - \gamma_k \sum_{i \in B_k}^n \nabla_\beta c_i(\mathbf{x}_i,
\mathbf{\beta}_k)
\]

which is the stochastic gradient descent method.

In linear regression the cost function is the mean square error. For the
gradient of the cost function we then have

\[
\nabla_{\beta} C(\mathbf{\beta}) = \frac{2}{n} \bigg( X^T (X\beta-y) + \lambda \beta \bigg)
\]

here I have added an \(L_2\) regularization parameter \(\lambda\) which
will be zero for ordinary least squares, but take some non-zero value
for ridge. The number of datapoints is given by n.~

For logistic regression the cost function is the cross-entropy, which we
went over in the section on logistic regression.

\hypertarget{decaying-learning-rate}{%
\subsubsection{Decaying Learning Rate}\label{decaying-learning-rate}}

Gradient methods can be sensitive to the choice of the step size or
learning rate \(\gamma\), and both too small and too large values can
give very poor results. One technique to limit this issue is to
gradually decrease the learning rate as we move through the epochs. The
idea is that we use large steps in the beginning when we are likely to
be far away from the solution, and smaller steps as we approach the
minimum.

The let the learning rate \(\gamma_j\) \emph{decay} according to the
function

\[
\gamma_j(t; t_0, t_1) = \frac{t_0}{t+t_1} 
\]

where \(t_0, t_1 > 0\) are fixed parameters, and \(t=i \cdot m + b\)
with \(i\) denoting the current epoch number, \(m\) the number of
mini-batches, and \(b\) the current batch number.

    \hypertarget{neural-network}{%
\subsection{Neural Network}\label{neural-network}}

Neural networks are a group of models originally inspired by biological
neuron used in supervised and unsupervised learning as well as
specialized tasks such as image processing. They are non-linear models
and can be considered powerful extensions of supervised learning methods
like linear and logistic regression We will be focusing on a type of
neuroal networks called feed-forward neural networks (FFNN).

    Figure 3: A schematic showing a fully connected FFNN with one hidden
layer.

    \hypertarget{feed-forward-neural-network}{%
\subsubsection{Feed-Forward Neural
Network}\label{feed-forward-neural-network}}

A FFNN is a type of artifical neural network (ANN) made up of layers of
connected neurons, also called nodes. They consist of an input layer,
followed by one or more so-called hidden layers, and finally an output
layer. In each layer there is a certain number of nodes, and this number
can vary between the layers. The nodes of one layer are connected with
the nodes of the next with an associated weight variable. In addition
there may be a bias in each layer. Training the model amounts to finding
optimal values for these weights and biases.

In FFNN the information flows only forward, from one layer to the next.
If all nodes in each layer are connected to all the nodes in the next we
have a fully connected network. According to the \emph{universal
approximation theorem}{[}4{]}, a FFNN with just a single hidden layer
containing a finite number of neurons can approximate a continuous
multidimensional function to arbitrary accuracy. This result assumes
that the activation function for the hidden layer is a non-constant,
bounded and monotonically-increasing continuous function.{[}5{]}

For a fully-connected model each input node sends its input \(x_j\) to
every node in the first hidden layer. The input to node \(i\) of the
first hidden layer becomes:


\begin{equation} z_i^1 = \sum_{j=1}^{M} w_{ij}^1 x_j + b_i^1
\end{equation}


    Each input is weighted bu \(w_{ij}\) and in addition to the sum over the
weighted inputs the node receives a bias contribution \(b_i^1\). This
bias is to assure we don't end up with zero activation in a layer, as
this would stop the flow of information from the input to the output,
and give us no output. The ouput from node \(i\) in the first hidden
layer is


\begin{equation}
 y_i^1 = f(z_i^1) = f\left(\sum_{j=1}^M w_{ij}^1 x_j  + b_i^1\right)
\end{equation}


where \(f(z)\) is the activation function for the hidden layer. The
output from the nodes in the hidden layer are given as weighted inputs
to all the nodes in the next hidden layer in the same way as described
here for this first layer, with a (different or equal) activation
function giving the ouput of that next layer. This continues until the
output layer. The output from this final layer is the model layer.
Typically the nodes in the hidden layers all have the same activation
function, while the ouput layer has a different one.

Generalizing we get the output from a model with \(l\) hidden layers as:


\begin{equation}
y^{l+1}_i = f^{l+1}\left[\!\sum_{j=1}^{N_l} w_{ij}^{l+1} f^l\left(\sum_{k=1}^{N_{l-1}}w_{jk}^{l}\left(\dots f^1\left(\sum_{m=1}^{M} w_{nm}^1 x_m+ b_n^1\right)\dots\right)+b_j^{l}\right)+b_i^{l+1}\right] 
\end{equation}


which is a nested sum of weighted activation functions.

With the biases and activations as \(N_l \times 1\) column vectors
\(\hat{b}_l\) and \(\hat{y}_l\), where the \(i\)-th element of each
vector is the bias \(b_i^l\) and activation \(y_i^l\) of node \(i\) in
layer \(l\) respectively, and the weights as an \(N_{l-1} \times N_l\)
matrix, \(\mathrm{W}_l\) we can write the sum as a matrix-vector
multiplication. Looking at hidden layer 2 for simplicity we can write
this in matrix notation as


\begin{equation}
 \hat{y}_2 =  
 f_2\left(\left[\begin{array}{cccc}
    w^2_{11} &w^2_{12}  &\cdots &w^2_{1N_l} \\
    w^2_{21} &w^2_{22} &\cdots &\vdots \\
    \vdots   &\vdots   &       &\vdots \\
    w^2_{N_{l-1}1} &w^2_{N_{l-1}2} &\cdots &w^2_{N_{l-1}N_l} \\
    \end{array} \right] \cdot
    \left[\begin{array}{c}
           y^1_1 \\
           y^1_2 \\
           \vdots \\
           y^1_{N_l} \\
          \end{array}\right] + 
    \left[\begin{array}{c}
           b^2_1 \\
           b^2_2 \\
           \vdots \\
           b^2_{N_l} \\
          \end{array}\right]\right).
\end{equation}


    In general we have an expression for \(z_i^l\), the activation of node
\(i\) of the \(l\)-th layer as

\[
z_i^l = \sum_{j=1}^{N_{l-1}}w_{ij}^la_j+b_i^l.
\]

Here \(b_i^l\) is the bias into node \(i\) in layer \(l\), \(w_{ij}^l\)
is the weight from node \(j\) in layer \(l-1\) on the input to node
\(i\) in layer \(l\), and \(a_j\) the output from node \(j\) in layer
\(l-1\).

The output from node \(i\) in layer \(l\) becomes

\[
a_i^l = f^l(z_i^l) = \frac{1}{1+\exp{-(z_i^l)}}.
\]

Using the sigmoid as activation function in layer \(l\).

    \hypertarget{regularization}{%
\subsection{Regularization}\label{regularization}}

We will be adding an \(L_2\) regularization parameter \(\lambda\) to our
logistic regression as well as our neural network code. See description
of ridge regression in project 1{[}1{]} for more info on \(L_2\)
regularization.

\hypertarget{l2-regularization-ffnn}{%
\subsubsection{L2 Regularization FFNN}\label{l2-regularization-ffnn}}

when adding L2 regularixation to a feed forward neural network with
backpropagation like we have, this amounts to adding a regularization
term to the cost function minimizing the size of the individual weights
in the model.

\[
C = C_0 + \frac{\lambda}{2n}\sum_w w^2
\]

where \(C_0\) is the original cost function without regularization and
\(\lambda\) is the regularization parameter. The regularization does not
have a dependency to the biases, so these are not affected, but the
derivative of the cost function with respect to the weights becomes

\[
\frac{\partial C}{\partial w} = \frac{\partial C_0}{\partial w} + \frac{\lambda}{n}w.
\]

Using this the weights are updated according to

\[
w -> w\bigg(1-\frac{\gamma\lambda}{n}\bigg)-\gamma\frac{\partial C_0}{\partial w}
\]

    \hypertarget{activation-function}{%
\subsubsection{Activation Function}\label{activation-function}}

A choice must be made for the activation function to be usen in the
nodes of the hidden layer as well as the nodes of the ouput layer. I
will be using the sigmoid function as well ass the ReLU and Leaky ReLU
in the hidden layers, and the softmax for the output layer when
classifying handwritten numbers. For regression with FFNN I will use
ReLU as the activation function for the output layer.

\hypertarget{sigmoid-function}{%
\paragraph{Sigmoid Function}\label{sigmoid-function}}

This activation function is popularly used in the ouput layer of binary
classification problems. This function is at risk for the so-called
vanishing gradient problem; that the gradient becomes too small for
effective model training. the function has the form

\[
f(x) = \frac{1}{1 + e^{-x}}.
\] 

A plot is seen in figure 4.

    \begin{center}
    \adjustimage{max size={0.9\linewidth}{0.9\paperheight}}{output_23_0.png}
    \end{center}
    { \hspace*{\fill} \\}
    
    \begin{Verbatim}[commandchars=\\\{\}]
Figure 4: Plot of the logistic sigmoid function
    \end{Verbatim}

    \hypertarget{hyperbolic-tangent-function}{%
\paragraph{Hyperbolic Tangent
Function}\label{hyperbolic-tangent-function}}

A mathematically shifted version of the sigmoid that generally performs
better than the sigmoid. It has the form

\[
f(x) = \tanh(x),
\] and a plot is seen in figure 5.


    \begin{center}
    \adjustimage{max size={0.9\linewidth}{0.9\paperheight}}{output_25_0.png}
    \end{center}
    { \hspace*{\fill} \\}
    
    \begin{Verbatim}[commandchars=\\\{\}]
Figure 5: Plot of the hyperbolic tangent function
    \end{Verbatim}

    \hypertarget{binary-step-function}{%
\paragraph{Binary Step Function}\label{binary-step-function}}

Also called the heaviside step function, or the unit step function. It
is an "all or nothing" approach and jumps from 0 to 1 at \(x=0\).

\[
f(x) =
\left\{
    \begin{array}{ll}
        1  & \mbox{if } x > 0 \\
        0 & \mbox{if } x \leq 0
    \end{array}
\right.
\] 
A plot is seen in figure 6.


    \begin{center}
    \adjustimage{max size={0.9\linewidth}{0.9\paperheight}}{output_27_0.png}
    \end{center}
    { \hspace*{\fill} \\}
    
    \begin{Verbatim}[commandchars=\\\{\}]
Figure 6: Plot of the binary step function
    \end{Verbatim}

    \hypertarget{relu-function}{%
\paragraph{ReLU Function}\label{relu-function}}

ReLU, or Rectified Linear Unit is a popular and fairly general
activation function. It is a common choice for the activation function
of the hidden layers. It is computationally cheap and therefore
efficient. One issue is that all negative values are mapped to zero,
which can lead to what is called \emph{dying ReLU problem}. This is an
issue that can occur if the weighted sum of a neuron's inputs is
negative. The neuron will then produce 0 as its output. Because the
gradient of the ReLU function is 0 when its input is negative this is
unlikely to change and the neuron is essentially "dead".

The rectifier is defined as the positive part of its argument

\[
f(x) = x^{+} = \max(0,x).
\] 

A plot is seen in figure 7.

    \begin{center}
    \adjustimage{max size={0.9\linewidth}{0.9\paperheight}}{output_29_0.png}
    \end{center}
    { \hspace*{\fill} \\}
    
    \begin{Verbatim}[commandchars=\\\{\}]
Figure 7: Plot of the ReLU function
    \end{Verbatim}

    \hypertarget{leaky-relu-function}{%
\paragraph{Leaky ReLU Function}\label{leaky-relu-function}}

Leaky ReLU solves the dying ReLU problem described in the last
subsection by assigning a small positive slope for \(x < 0\). This does
however reduce the performance/computational cost.

\[
f(x) =
\left\{
    \begin{array}{ll}
        x  & \mbox{if } x \geq 0 \\
        0.01x & \mbox{otherwise }
    \end{array}
\right.
\]

See figure 8 for a plot of this function.

    \begin{center}
    \adjustimage{max size={0.9\linewidth}{0.9\paperheight}}{output_31_0.png}
    \end{center}
    { \hspace*{\fill} \\}
    
    \begin{Verbatim}[commandchars=\\\{\}]
Figure 8: Plot of the Leaky ReLU function. It is a bit hard to spot, but the
slope from x=-10 to x=0 is positive.
    \end{Verbatim}

    \hypertarget{elu-function}{%
\paragraph{ELU Function}\label{elu-function}}

The ELU or exponential linear unit function is another function in the
ReLU family proposed to avoid the \emph{dying ReLU problem}. It does
however add another parameter to the model \(\alpha\).

\[
f(x) =
\left\{
    \begin{array}{ll}
        \alpha(\exp(x)-1  & \mbox{if } x < 0 \\
        x & \mbox{otherwise }
    \end{array}
\right.
\] 
A plot is seen in figure 9.

    \begin{center}
    \adjustimage{max size={0.9\linewidth}{0.9\paperheight}}{output_33_0.png}
    \end{center}
    { \hspace*{\fill} \\}
    
    \begin{Verbatim}[commandchars=\\\{\}]
Figure 9: Plot of the ELU function. I have set the paramater alpha equal to 1.
    \end{Verbatim}

    \hypertarget{softmax-function}{%
\paragraph{Softmax Function}\label{softmax-function}}

A generalization of the logistic function often used in multinomial
logistic regression.

\[
f(x) = \frac{e^x}{\sum_{j=1}^K e^{x}},
\]

The function takes as input a vector \(x\) of \(K\) real numbers, and
normalizes it into a probability distribution consisting of \(K\)
probabilities proportional to the exponentials of the input numbers. See
figure 10 for a plot.

    \begin{center}
    \adjustimage{max size={0.9\linewidth}{0.9\paperheight}}{output_35_0.png}
    \end{center}
    { \hspace*{\fill} \\}
    
    \begin{Verbatim}[commandchars=\\\{\}]
Figure 10: Plot of the softmax function.
    \end{Verbatim}

    \hypertarget{results-and-discussion}{%
\section{Results and Discussion}\label{results-and-discussion}}

\hypertarget{linear-regression-with-gradient-descent}{%
\subsection{Linear Regression with Gradient
Descent}\label{linear-regression-with-gradient-descent}}

In project one we looked modeled the Franke function using ordinary
least squares (OLS) and ridge regression. Now we want to compare these
methods to methods using stochastic gradient descent to obtain the fit,
i.e the parameters or weights \(\beta\).

The comparison is done on data generated with the Franke function, with
added stochastic noise with \(\sigma^2=0.1\). See figure 11 for a 3d
plot of these data.

    \begin{center}
    \adjustimage{max size={0.9\linewidth}{0.9\paperheight}}{output_37_0.png}
    \end{center}
    { \hspace*{\fill} \\}
    
    \begin{Verbatim}[commandchars=\\\{\}]
Figure 11: 3D plot of the Franke function with added noise.
    \end{Verbatim}

    \begin{Verbatim}[commandchars=\\\{\}]
** Comparing my ols implementation with that of scikit-learn: **
The arrays are about the same:  True
The mean squared error between my result and that of scikit-learn is:
2.0601779122756744e-22
    \end{Verbatim}

    We fit our regression models to this data. For OLS there are no
parameters to tune. We find the mean square error and the \(r^2\) score
of our OLS method, see table 1.


    \begin{Verbatim}[commandchars=\\\{\}]
Table 1: Error measures fitting OLS method to noisy Franke data.
*** Error measures for OLS method: ***
Train:   MSE:  0.012086801248348334   r2:  0.8644398531166579
Test:    MSE:  0.012501297568858788   r2:  0.8583664210243516
    \end{Verbatim}

    For linear regression with gradient descent we have several parameters,
the learning rate, the batch size, and the number of epochs. We explore
how the error is affected by various choices for these parameters. The
results for learning rate can be seen in figure 12. We can see that the
error improves steadily, although somewhat erraticaly as the learning
rate increases before exploding when we reach a too large learning rate.
Too large is seen to be about \(\gamma>0.07\) .



    \begin{center}
    \adjustimage{max size={0.9\linewidth}{0.9\paperheight}}{output_44_0.png}
    \end{center}
    { \hspace*{\fill} \\}
    
    \begin{Verbatim}[commandchars=\\\{\}]
Figure 12: The mean square error on the test set as the learning rate of the SGD
method is increased.
No regularization.
    \end{Verbatim}

    The increased error for very small learning rate can be explained by the
SGD method taking such small steps down the slope it does not reach a
minimum, whereas with a too large learning rate the minimum is skipped
altogether.

We can let the learning rate vary according to a learning schedule as
described in the methods section on stochastic gradient descent. The
results are plotted in figure 13.

    \begin{center}
    \adjustimage{max size={0.9\linewidth}{0.9\paperheight}}{output_47_0.png}
    \end{center}
    { \hspace*{\fill} \\}
    
    \begin{Verbatim}[commandchars=\\\{\}]
Figure 13: The mean square error for varying learning schedule.
    \end{Verbatim}

    We see that a too small value for \(\texttt{t1}\) leads to the error
exploding, but the error also startes rising for large t1. The sweet
spot appears to be around \(50<\texttt{t1}<200\).

We also want to explore how the error is affected by the number of
epochs used. See figure 14 for a plot. As expected the error decreases
as the number of epochs increases. At first there is a steep
improvement, but as we reach a few hundred epochs the rate of
improvement has slowed down markedly and we gain very little relative to
the increase in computational cost. Although there is some benefit to
increasing to 1000+ epochs, whether that is worth it or not will depend
on computation cost and precision requirements.

    \begin{center}
    \adjustimage{max size={0.9\linewidth}{0.9\paperheight}}{output_50_0.png}
    \end{center}
    { \hspace*{\fill} \\}
    
    \begin{Verbatim}[commandchars=\\\{\}]
Figure 14: The mean square error for increasing number of epochs.
    \end{Verbatim}

    \begin{center}
    \adjustimage{max size={0.9\linewidth}{0.9\paperheight}}{output_52_0.png}
    \end{center}
    { \hspace*{\fill} \\}
    
    \begin{Verbatim}[commandchars=\\\{\}]
Figure 15: The mean square error for increasing the batch size.
    \end{Verbatim}

    A similar plot for the choice of batchsize can be seen in figure 15. As
we see the error explodes for small batch size, but for any batchsize
\textgreater{} 7 or so the performance is fairly stable at a minimum. As
the batchsize grows very big the error starts to climb up, but even at
about 1/3 of the size of the training set the error is small. The reason
the error starts increasing here is that the number of epochs is
constant, so with very large batchsize we perform fewer steps per epoch
and the increased precision of the gradient calculation does not make up
for this reduction in steps. For very small batchsizes the estimate for
the gradient is simply too inaccurate, and we are not guaranteed to be
moving down the gradient slope. We see that the error increase is slight
for the batchsized explored, but one should be careful to choose a very
big batchsize.

    We move on to look at ridge regression with gradient descent. I will be
using k-fold cross validation to explore choices for the regression
parameter \(\lambda\) and how this choice affects model performance.
Figure 16 shows a heatmap of the results.

        
    \begin{center}
    \adjustimage{max size={0.9\linewidth}{0.9\paperheight}}{output_56_1.png}
    \end{center}
    { \hspace*{\fill} \\}
    
    \begin{Verbatim}[commandchars=\\\{\}]
Figure 16: A heatmap to illustrate how the mean square error of the ridge
regression model with
stochastic gradient descent depends on the choice of ridge parameter lambda and
learning rate gamma.
    \end{Verbatim}

    We see from figure 16 that for a small enough regression parameter
\(\lambda\), the MSE is fairly robust to the choice of the learning rate
\(\gamma\). As \(\lambda\) grows however, the range of \(\gamma\)-values
giving a small MSE narrows.

    Now that we have found some optimal parameters for SGD on this
particular problem, let us compare our model performance with that of
scikit-learn's \(\texttt{SGDRegressor}\). A table of the results is seen
in table 2.

    Table 2: Comparing my own ridge regression model with SGD with
\(\texttt{SGDRegressor}\) from \(\texttt{scikit-learn}\).

    \begin{Verbatim}[commandchars=\\\{\}]
SGDRegressor: mse=0.0308    r2=0.6509
Own ridge SGD: mse=0.0164    r2=0.8146
    \end{Verbatim}

    I see the error is actually better for my implementation using these
parameter, although this method is noticeably slower. The improved error
and at least part of the speed improvement is likely due to
\(\texttt{SGDRegressor}\) calculating the gradient on only one datapoint
at a time, instead of a batch{[}3{]}. This gives a noisier/more erratic
movement towards the minimum.

    \hypertarget{neural-network---regression}{%
\subsubsection{Neural Network -
Regression}\label{neural-network---regression}}

The second of the overarching goals in this project was to use a feed
forward neural network with backpropagation (FFNN) to model the Franke
function. We have used the sigmoid function as activation function for
the hidden layers, while the output layer has no activation function as
I want the actual, continuous values, i.e.~not translated into a binary
yes/no value or a set of classes. Cross-entropy is used as cost
function.

The network is set up with a number of input nodes matching the number
of terms in the design matrix \(\texttt{X}\), and just one output node
as we only have one output, z.

I try out a few different configurations for the hidden layer(s). Figure
17 shows a plot of the resulting \(r^2\) scores, all using 100 epochs
and a batchsixe of 50. The configurations correspond to the following
set up: * config1 = {[}21,1{]} * config2 = {[}21,5,1{]} * config3 =
{[}21,10,1{]} * config4 = {[}21,5,2,1{]} * config5 = {[}21,5,5,1{]} *
config6 = {[}21,10,5,1{]} * config7 = {[}21,10,10,1{]} * config8 =
{[}21,15,10,5,1{]}

which shows a list of the number of nodes per layer from input layer,
via all the hidden layers, ending with the output layer. The model is
tested on a set of learning rates, but the learning rate where the error
explodes is different for the various configurations. I have plotted
only the sensible values, so when the lines in figure 17 stop that means
their error exploded for the next value of the learning rate \(\gamma\).

    \begin{center}
    \adjustimage{max size={0.9\linewidth}{0.9\paperheight}}{output_65_7.png}
    \end{center}
    { \hspace*{\fill} \\}
    
    Figure 17: \(R^2\) score on test set for a selection of architectures
for the neural network. For all models the logistic sigmoid is used in
the hidden layers, cross entropy is the cost function. The models are
fitted over 100 epochs with a batchsize of 50.

    Of the network architectures tried, the optimal seems to be
configuration number 6 with two hidden layers, one with 10 nodes and one
with 5. We will use this configuration from here on.

We do, however, see that performance is fairly stable for the majority
of the tested configurations. Configuration 1 (no hidden layers) seems to be
particularly sensitive to the choice of learning rate, but apart from
this they are all in agreement on the optimal range for the learning
rate. In addition the optimal choice of the learning parameter gives
fairly similar values for the \(r^2\) score. It is worth noting that the
parameters such as number of epochs and batch size has not been
exhaustively explored for all the configurations shown. As such some are
likely to have even better `optimal' \(r^2\) scores for different
hyper-parameters.

    \begin{center}
    \adjustimage{max size={0.9\linewidth}{0.9\paperheight}}{output_69_0.png}
    \end{center}
    { \hspace*{\fill} \\}
   
    \begin{Verbatim}[commandchars=\\\{\}]
Figure 18: How the performance of the FFNN regression model on the franke data
changes with the learning rate.
One hidden layer with 10 nodes, and one with 5. The batchsize is 50 and the
number of epochs is 100. The hidden
layer uses the sigmoid activation function and the output layer uses no
activation function.
    \end{Verbatim}

    \begin{center}
    \adjustimage{max size={0.9\linewidth}{0.9\paperheight}}{output_72_0.png}
    \end{center}
    { \hspace*{\fill} \\}
    
    \begin{Verbatim}[commandchars=\\\{\}]
Figure 19: How the performance of the FFNN regression model on the franke data
changes with the number of
epochs. One hidden layer with 10 nodes, one with 5. The batchsize is 100 and the
learning rate is 0.8. The
hidden layer uses the sigmoid activation function and the output layer uses no
activation function.
    \end{Verbatim}

    Figure 18 shows the \(r^2\) score as a function of the number of epochs
using configuration 6. We see the \(r^2\) score improves with increasing
the number, as expected, but from 100 epochs and on it flattens out and
no further improvement is seen, meaning increasing the number above this
will increase the computational cost at minimal to no additional gain in
performance.

Next up is the batch size, as seen in the plot in figure 19. Even for
small batch sizes the performance here is good, we don't see the same
sensitivity we saw using linear regression with SGD, but we get a
similar dip for very large batch sizes. Note that the training set has
7500 entries, so a batch size of 500 (the first data point where we see a
reduction in the r2 score) is about \(7\%\) of the total training set.

    \begin{center}
    \adjustimage{max size={0.9\linewidth}{0.9\paperheight}}{output_75_0.png}
    \end{center}
    { \hspace*{\fill} \\}
    
    \begin{Verbatim}[commandchars=\\\{\}]
Figure 19: How the performance of the FFNN regression model on the franke data
changes with the size of
the batches used in the stochastic gradient descent. One hidden layer with 10
nodes, one with 5. The number of epochs
is 100 and the learning rate is 0.8. The hidden layer uses the sigmoid
activation function and the output layer uses
no activation function.
    \end{Verbatim}

    \hypertarget{adding-l2-regularization}{%
\paragraph{Adding L2 Regularization}\label{adding-l2-regularization}}

We now explore adding regularization to the method, namely l2
regularization. First we look at the choice of the regularization
parameter \(\lambda\) searching for an optimized value. Figure 21 shows
a plot.

    \begin{center}
    \adjustimage{max size={0.9\linewidth}{0.9\paperheight}}{output_78_0.png}
    \end{center}
    { \hspace*{\fill} \\}
    
    \begin{Verbatim}[commandchars=\\\{\}]
Figure 21: How the performance of the FFNN regression model on the franke data
changes with the strength/weight
of the l2 regularization. Model with one hidden layer with 10 nodes, one with 5.
The number of epochs is 100 and
the learning rate is 0.8. The hidden layer uses the sigmoid activation function
and the output layer uses none.
    \end{Verbatim}

    \hypertarget{different-activation-functions}{%
\paragraph{Different Activation
Functions}\label{different-activation-functions}}

There are several available options for the activation function of the
nodes in the hidden layers. Figure 22 shows a plot of the r2 score for
models when the shape of this activation function is altered. Again the
error explodes at different values for the learning rate for the
different activation functions, and we have avoided plotting the
exploding scores, which is why the plots are over differing ranges of
the learning rate. We see from figure 22 that ReLU, ELU and the sigmoid
all give similar performance, while leaky ReLU seems to struggle more
with the learning rate. The sigmoid gives the best \(r^2\) score for the
chosen parameters, but ReLU is seen to be somewhat more robust to the
choice of learning rate.

    \begin{center}
    \adjustimage{max size={0.9\linewidth}{0.9\paperheight}}{output_81_3.png}
    \end{center}
    { \hspace*{\fill} \\}
    
    Figure 22: R2 score on test set for a selection of activation functions
used in the hidden layers of the neural network. For all models cross
entropy is the cost function and the network architecture is as in
config6. The models are fitted over 100 epochs with a batch size of 50

    Now that we have optimized our parameters to an extent, we compare the
performance of this model with that of the \(\texttt{MLPRegressor}\) in
\(\texttt{scikit-learn}\). The results can be seen in table 3. For both
models we used a learning rate of \(0.75\), a batch size of \(50\), and
\(100\) epochs. We see that the resulting errors are almost identical.

    Table 3: Comparing my own FFNN regression model with
\(\texttt{MLPRegressor}\) from \(\texttt{scikit-learn}\). Both use with
SGD and the logistic sigmoid function as activation function for the
hidden layers.


    \begin{Verbatim}[commandchars=\\\{\}]
Own FFNN Regressor with SGD:                 mse=0.0110    r2=0.8752
MLPRegressor from scikit-learn with SGD:     mse=0.0116    r2=0.8691
    \end{Verbatim}

    \hypertarget{neural-network---classification}{%
\subsubsection{Neural Network -
Classification}\label{neural-network---classification}}

We move on to look at using our neural network for classification. We
use the FFNN model to perform classification of handwritten numbers.
First we classify the numbers using \(\texttt{scikit-learn}\)'s
\(\texttt{MLPClassifier}\) to have something to compare the next results
to. The network is set up with 64 input nodes, corresponding to the 64
pixels in each digit image, one layer of 30 hidden nodes, and finally an
output layer of 10 nodes, corresponding to the 10 output possibilities.
The activation function of the output layer is the softmax, while the
hidden layer(s) use the sigmoid. We are not using regularization here.
Note that the \(\texttt{MLPClassifier}\) implicitly designs the input
and output layers based on the provided data in \(\texttt{fit()}\)
method. It uses cross-entropy as the default cost function.

    \begin{Verbatim}[commandchars=\\\{\}]
The measured accuracy of FFNN on the test set using scikit-learn's MLPClassifier
is 0.9778
    \end{Verbatim}

    We get a very impressive accuracy score of \(\approx0.98\) using
\(\texttt{scikit-learn}\). We compare this with our own code which gives
a very similar result of \(\approx0.99\) using 30 epochs and a learning
rate of 0.5.

    \begin{Verbatim}[commandchars=\\\{\}]
The measured accuracy of FFNN on the test set using our nown NeuralNetClassifier
with one hidden layer of 30 nodes, softmax activation function in the output
layer
and sigmoid in the hidden layer is: 0.9889
    \end{Verbatim}

    Like for regression we are interested in exploring a few different
configuartions or network architectures. Figure 23 shows a plot of how
the resulting accuracy for a selection of configurations varies with the
choice of learning rate. The tested configurations are:

\begin{itemize}
\tightlist
\item
  config1 = {[}64,10{]}
\item
  config2 = {[}64,10,10{]}
\item
  config3 = {[}64,30,10{]}
\item
  config4 = {[}64,10,10,10{]}
\item
  config5 = {[}64,30,30,10{]}
\item
  config6 = {[}64,50,40,30,10{]}
\item
  config7 = {[}64,50,40,30,20,10{]}
\end{itemize}

We see from the plot that for all the configuration the general behavior
is the same. The performance improves with the increasing learning rate
up to a point. The differences relate to how high the accuracy gets, and
how soon it responds to the increase in learning rate. The drop off
learning rate is similar for all the architectures. Even configuration 1
performs really well, despite not having any hidden layers. It is
however the most sensitive to too high learning rate. In contrast it
seemingly handles small learning rates better than the others. Based on
figure 23 configuration 3 seems a solid choice, but any of configuration 1, 2, 3
and 5 display a satisfactory performance and stability.


    \begin{center}
    \adjustimage{max size={0.9\linewidth}{0.9\paperheight}}{output_94_3.png}
    \end{center}
    { \hspace*{\fill} \\}
    
    Figure 23: A plot showing how the model accuracy of our FFNN model for
classification when identifying handwritten digits for a selection of
network architectures. Batch size is 100 and the number of epochs is 30.


    \begin{center}
    \adjustimage{max size={0.9\linewidth}{0.9\paperheight}}{output_98_0.png}
    \end{center}
    { \hspace*{\fill} \\}
    
    \begin{Verbatim}[commandchars=\\\{\}]
Figure 24: How accuracy of the FFNN classification model changes with the number
of epochs. One hidden layer with 30 nodes, batchsize is 100 and the learning
rate is 0.5. The hidden layer uses the sigmoid activation function and the
output layer uses the softmax.
    \end{Verbatim}

    \begin{Verbatim}[commandchars=\\\{\}]
The max accuracy on the test set is 0.994444 after 25 epochs.
    \end{Verbatim}


    \begin{center}
    \adjustimage{max size={0.9\linewidth}{0.9\paperheight}}{output_101_0.png}
    \end{center}
    { \hspace*{\fill} \\}
    
    \begin{Verbatim}[commandchars=\\\{\}]
Figure 25: How accuracy of the FFNN classification model changes with the
learning rate. One hidden layer with 30 nodes, batchsize is 100 and the number
of epochs is 100. The hidden layer uses the sigmoid activation function and the
output layer uses the softmax.
    \end{Verbatim}


    \begin{center}
    \adjustimage{max size={0.9\linewidth}{0.9\paperheight}}{output_103_0.png}
    \end{center}
    { \hspace*{\fill} \\}
    
    \begin{Verbatim}[commandchars=\\\{\}]
Figure 26: How accuracy of the FFNN classification model changes with the
batch size. One hidden layer with 30 nodes, larning rate is 0.5 and the number
of epochs is 100. The hidden layer uses the sigmoid activation function and the
output layer uses the softmax.
    \end{Verbatim}

    \hypertarget{adding-regularization}{%
\paragraph{Adding Regularization}\label{adding-regularization}}

To avoid over-fitting and increase the generalization of a model one can
add a regularization term. As we have seen in the results so far we
don't appear too troubled by over-fitting in our classification model
with our current parameters. Adding regularization can also make the
model more stable to changes in these parameters.

    \begin{center}
    \adjustimage{max size={0.9\linewidth}{0.9\paperheight}}{output_106_0.png}
    \end{center}
    { \hspace*{\fill} \\}
    
    \begin{Verbatim}[commandchars=\\\{\}]
Figure 27: The FFNN classification model with added l2 regularization. One
hidden layer with 30 nodes, learning
rate is 0.5 and the number of epochs is 100. The hidden layer uses the sigmoid
activation function and the output
layer uses the softmax.
    \end{Verbatim}

    \hypertarget{logistic-regression---classification}{%
\subsubsection{Logistic Regression -
Classification}\label{logistic-regression---classification}}

So far we have used a FFNN neural network for classification. We want to
compare our results with a different method. The method of choice is
logistic regression. As we are performing classification to multiple
classes (not binary classification) we use the softmax in place of the
standard sigmoid. This form of logistic regression is called multinomial
logistic regression or simply softmax regression. Our cost function is
again cross-entropy.

We have our own code for this as well, but first we attempt
classification of our digit data with \(\texttt{scikit-learn}\)'s
\(\texttt{SGDClassifier}\). Setting the \(\texttt{loss}\) parameter of
this method to `log' gives a logistic regression for classification with
SGD as training method.

When fitting the \(\texttt{SGDClassifier}\)-model to the training set of
the digits data and using the fitted model to predict the digits based
on the input values of the test set we get an accuracy score of
\(0.95\). This is exactly the same score that we achieve from our own
logistic regression code for multinomial regression with stochastic
gradient descent. For both models the learning rate was \(0.1\).

    \begin{Verbatim}[commandchars=\\\{\}]
Accuracy when using SGDClassifier with 'log'-loss on digits-data is: 0.95
    \end{Verbatim}

    \begin{Verbatim}[commandchars=\\\{\}]
Accuracy when using own code for logistic regression with SGD on digits-data is:
0.95
    \end{Verbatim}

    Figure number shows a plot of how the accuracy after on the test set
varies with the chosen learning rate. We see a similar behavior as
before, with very small and very large learning rates resulting in a
poor accuracy score. While the performance of the two methods is very
similar in the middle range of the learning rates it is interesting to
see that \(\texttt{SGDClassifier}\) seems to be less sensitive to the
choice of learning rate.


    \begin{center}
    \adjustimage{max size={0.9\linewidth}{0.9\paperheight}}{output_112_0.png}
    \end{center}
    { \hspace*{\fill} \\}
 

    \begin{Verbatim}[commandchars=\\\{\}]
Figure 28: Accuracy on the test set of the digits dataset for varying
learning rate on multinomial
logistic regression with stochastic gradient descent when the methods are fitted
to the corresponding training set.
    \end{Verbatim}



    \begin{center}
    \adjustimage{max size={0.9\linewidth}{0.9\paperheight}}{output_116_0.png}
    \end{center}
    { \hspace*{\fill} \\}
    
    Figure number: The FFNN classification model with added l2
regularization. One hidden layer with 30 nodes, learning rate is 0.5 and
the number of epochs is 100. The hidden layer uses the sigmoid
activation function and the output layer uses the softmax.

    \hypertarget{conclusion}{%
\section{Conclusion}\label{conclusion}}

    \begin{tcolorbox}[breakable, size=fbox, boxrule=1pt, pad at break*=1mm,colback=cellbackground, colframe=cellborder]
\prompt{In}{incolor}{ }{\boxspacing}
\begin{Verbatim}[commandchars=\\\{\}]

\end{Verbatim}
\end{tcolorbox}

    \hypertarget{bibliography}{%
\section{Bibliography}\label{bibliography}}

{[}1{]} Project 1:
https://github.com/emiliefj/FYS-STK3155/blob/master/Project1/Report/Project\%201\%20-\%20FYS-STK3155.pdf

{[}2{]} Digits dataset in scikit-learn:
https://scikit-learn.org/stable/auto\_examples/datasets/plot\_digits\_last\_image.html

{[}3\}
https://www.bogotobogo.com/python/scikit-learn/scikit-learn\_batch-gradient-descent-versus-stochastic-gradient-descent.php

{[}4{]} Universal approximation theorem:
https://en.wikipedia.org/wiki/Universal\_approximation\_theorem

{[}5{]} Slides week 40:
https://compphysics.github.io/MachineLearning/doc/pub/week40/html/week40.html

{[}6{]} Softmax function:
https://en.wikipedia.org/wiki/Softmax\_function

{[}7{]} Neural Networks and classification of handwritten digits:
http://neuralnetworksanddeeplearning.com

{[}8{]} Softmax Regression tutorial:
http://saitcelebi.com/tut/output/part2.html

{[}9{]} Standford Softmax Regression tutorial:
http://deeplearning.stanford.edu/tutorial/supervised/SoftmaxRegression/
    
    
    
\end{document}
